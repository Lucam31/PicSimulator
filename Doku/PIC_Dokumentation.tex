\documentclass[a4paper,11pt]{report}

\usepackage{Doku}

\begin{document}

\begin{titlepage}
    \centering
    \vspace*{2cm}
    
    \vspace{1.5cm}
    {\Huge\bfseries\textcolor{primarycolor}{PIC Simulator} \par}
    \vspace{1cm}
    {\Large\textbf{Software-Dokumentation} \par}
    \vspace{3cm}

    \begin{tabular}{l@{\hspace{2cm}}l}
        Projektnamen:	         & PIC Simulator \\
        Datum:	                 & 12.05.2025 \\
    \end{tabular}
    
    \vspace{3cm}
    
    \begin{tabular}{l@{\hspace{2cm}}l}
        Autoren:	         & Luca Müller \\
        	                 & Leander Gantert \\
    \end{tabular}
    
    
    
\end{titlepage}

\chapter{Einleitung}

\section{Über dieses Dokument}
Dieses Dokument beschreibt den PIC Simulator, seine Funktionen, Implementierung und Nutzung. Die Dokumentation richtet sich an Benutzer und Entwickler, die den Simulator verstehen und erweitern möchten.

\section{Projektziele}
Der PIC Simulator wurde entwickelt, um [Ziele des Projekts hier einfügen]. Die Hauptfunktionen umfassen:

\begin{itemize}
    \item Simulation von PIC-Mikrocontroller-Programmen
    \item Visualisierung des Programmablaufs
    \item Debugging-Funktionalitäten
    \item [Weitere Funktionen hier ergänzen]
\end{itemize}

\chapter{Systemarchitektur}

\section{Überblick}
Die Architektur des PIC Simulators basiert auf [Architekturprinzipien beschreiben]. Das System besteht aus folgenden Hauptkomponenten:

\begin{itemize}
    \item Kernkomponente: Verantwortlich für die Simulation des PIC-Prozessors
    \item Benutzeroberfläche: Ermöglicht die Interaktion mit dem Simulator
    \item Speichermodell: Simuliert den Speicher des Mikrocontrollers
    \item [Weitere Komponenten hier ergänzen]
\end{itemize}

\section{Komponenten-Diagramm}
[Hier könnte ein UML-Diagramm oder eine andere graphische Darstellung der Systemarchitektur eingefügt werden.]

\end{document}